\documentclass[12pt, a4paper]{article}
\usepackage{bookmark}
\usepackage{hyperref}

\title{Bibliography in LaTeX}
\author{Pulakesh Pradhan}
\date{\today}

\begin{document}

\maketitle

\section{Introduction}
Citing sources is crucial in academic writing. LaTeX simplifies this process.

\section{Citations}
As shown by Einstein \cite{einstein1905}, $E=mc^2$ is fundamental.
Also, check Knuth's work on TeX \cite{knuth1984}.

\section{Custom Bibliography}
For simpler documents, you can manually create the bibliography environment.
(For larger projects, use BibTeX or BibLaTeX).

\begin{thebibliography}{9}

\bibitem{einstein1905}
  Albert Einstein,
  \textit{Zur Elektrodynamik bewegter K{\"o}rper}. (German)
  [On the electrodynamics of moving bodies],
  Annalen der Physik,
  1905.

\bibitem{knuth1984}
  Donald E. Knuth,
  \textit{The \TeX book},
  Addison-Wesley, Reading, Massachusetts,
  1984.

\end{thebibliography}

\end{document}
