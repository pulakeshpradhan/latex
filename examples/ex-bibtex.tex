\documentclass[12pt, a4paper]{article}
\usepackage{bookmark}
\usepackage{hyperref}

\title{BibTeX Citation Example}
\author{Pulakesh Pradhan}
\date{\today}

\begin{document}

\maketitle

\section{Introduction}
Using a separate \texttt{.bib} file is the standard way to manage references in LaTeX. This allows you to reuse the same database across multiple documents.

\section{Citations}
\subsection{Classic Citations}
One of the most famous equations is $E=mc^2$, proposed by Einstein \cite{einstein1905}.
For typesetting, we owe everything to Donald Knuth \cite{knuth1984}.
Michel Goossens \cite{latexcompanion} wrote the definitive guide to \LaTeX.

\subsection{Online Citation}
This guide was created by Pradhan \cite{mkdocs}.

\section{How It Works}
\begin{enumerate}
    \item Create a \texttt{.bib} file (e.g., \texttt{references.bib}).
    \item Use \texttt{\textbackslash cite\{key\}} in your \texttt{.tex} file.
    \item Compile: \texttt{pdflatex} $\rightarrow$ \texttt{bibtex} $\rightarrow$ \texttt{pdflatex} $\rightarrow$ \texttt{pdflatex}.
\end{enumerate}

\bibliographystyle{plain} % plain, unsrt, alpha, abbrv, ieeetr
\bibliography{references} % The name of your .bib file (without extensions)

\end{document}
