\documentclass[12pt, a4paper]{article}
\usepackage[utf8]{inputenc}
\usepackage{geometry}
\geometry{margin=1in}
\usepackage{enumitem}
\usepackage{hyperref}

\title{LaTeX Basics Example}
\author{Pulakesh Pradhan}
\date{\today}

\begin{document}

\maketitle
\tableofcontents
\newpage

\section{Introduction}
This document demonstrates basic text formatting and document structure in LaTeX.

\subsection{Text Formatting}
We can make text \textbf{bold}, \textit{italic}, \underline{underlined}, or use \texttt{typewriter font}.
We can also combine them: \textbf{\textit{Bold and Italic}}.

Font sizes can vary:
\begin{itemize}
    \item \tiny This is tiny.
    \item \scriptsize This is scriptsize.
    \item \small This is small.
    \item \normalsize This is normal.
    \item \large This is large.
    \item \Huge This is huge.
\end{itemize}

\subsection{Alignment}
\begin{center}
    This text is centered.
\end{center}

\begin{flushright}
    This text is right-aligned.
\end{flushright}

\section{Lists}

\subsection{Bulleted List}
\begin{itemize}
    \item Geometric Shapes
    \begin{itemize}
        \item Circle
        \item Square
        \item Triangle
    \end{itemize}
    \item Colors
    \item Numbers
\end{itemize}

\subsection{Numbered List}
\begin{enumerate}
    \item First Step: Install LaTeX
    \item Second Step: Choose an Editor
    \item Third Step: write Code
    \item Fourth Step: Compile
\end{enumerate}

\subsection{Description List}
\begin{description}
    \item[LaTeX] A high-quality typesetting system.
    \item[Editor] Software to write your code (e.g., VS Code).
    \item[PDF] The final output format.
\end{description}

\section{Conclusion}
This concludes the basic formatting demonstration.

\end{document}
